\documentclass{article}

\usepackage[utf8]{inputenc}
\usepackage[english]{babel}
\usepackage[varg]{txfonts}
\usepackage{amsthm}
\usepackage{color}
\usepackage{gensymb}
\usepackage{bm}
\usepackage{graphicx}
% \usepackage{amssymb}
% % \usepackage{epsfig}
% \usepackage{graphics}
% \usepackage{amsmath}
% \usepackage{natbib}
% \usepackage{hyperref}
% \usepackage{float}

% \usepackage{mathtools}
% \usepackage{lipsum}

\newcommand{\be}{\begin{equation}}
\newcommand{\ee}{\end{equation}}
\def\bc{\begin{center}}
\def\ec{\end{center}}
\newcommand*\df {\mathop{}\!\mathrm{d}}
\def\beq{\begin{eqnarray}}
\def\eeq{\end{eqnarray}}
\newcommand{\red}[1]{\textcolor{red}{#1}}
\newcommand{\blue}[1]{\textcolor{blue}{#1}}
\newcommand{\green}[1]{\textcolor{green}{#1}}

\theoremstyle{plain}
\newtheorem{theorem}{Theorem}[section]
\theoremstyle{definition}
\newtheorem{definition}{Definition}[section]

\newcommand{\cmz}{\mathbb C \setminus \{0\} }
\newcommand{\pir}{\pi \mathbb Q }
\newcommand{\pirr}{\mathbb R \setminus \pi \mathbb Q }



\begin{document}
\bc
\Huge
	Elementary theory of the elemental roots.
\ec


\section{Notation}
A complex number $c \in \cmz $ can be expressed it as $$
  c = x_c + iy_c = \rho_c e^{i\varphi_c}, \qquad x,y,\rho,\varphi \in \mathbb R, \; \rho > 0, \; 0 \le \varphi < 2\pi
$$
\section{The fundamental sequence}

\begin{definition}
We call the \red{fundamental} sequence  $a_n$ of the \blue{whorl} $s \in \cmz$,  \blue{frond} $t \in \cmz$ and \blue{torsion} $\mu \in \mathbb R$ such a sequence that $a_0=1$ and $a_n = a_{n-1}r_n$ for $n\in \mathbb N$, where the series of factors $r_n = s + 2t \sin (\mu n)$ is called the \blue{T-series}.   
\end{definition}

\begin{definition}
The torsion is called a \red{jerk torsion} if $\mu \in \pir$. Otherwise, in case of $\pi$-irrational torsion $\mu \in \pirr$, it is called a \blue{sleek torsion}. 
\end{definition} 

\begin{definition}
The frond is classified by its absolute value, it can be either \blue{cramped} ($0< \rho_t<1$), \blue{unit} ($\rho_t=1$) or \blue{fancy} ($\rho_t>1$).
\end{definition} 

\begin{definition}\label{def:brim}
A \blue{brim} set $B(t)$ for fixed frond $t$ is such a set of whorls that $$
s \in B(t) \Leftrightarrow \exists \limsup_{n \to \infty}(\rho_{a_n})>\liminf_{n \to \infty}(\rho_{a_n})>0
.$$  
\end{definition} 

\begin{theorem}
If the jerk torsion $\mu = \frac{2\pi M}{N}$ for some integers $M \in \mathbb Z$ and $N \in \mathbb N$ then the brim consists of roots of the polynomial$$
\prod_{n=1}^{N} \rho_{r_n} - 1
$$
% , or in other words
% $$s \in B(t) \Leftrightarrow  
% \sum_{n=1}^{N} \ln (\rho_{r_n}) = 0$$
\end{theorem}
\begin{proof}
This is obvious.%, e.g. see the next theorem (Th. \ref{th:intbrim}).
\end{proof}

\begin{theorem}\label{th:intbrim}
In case of a sleek torsion
$$s \in B(t) \Leftrightarrow 
\int_0^{2\pi} \ln (\rho_{r_1}) d\mu = 0 .$$
\end{theorem}
\begin{proof}
The condition for $s$ to be in the brim (Def. \ref{def:brim}) can be written as \be\label{eq:brim}
s \in B(t) \Leftrightarrow \exists  l>0 \in \mathbb R : \forall N \in \mathbb N : e^{-l}<\rho_{ a_N}  < e^l
.\ee
Taking logarithm of the last inequality one gets $$
l>\left|\ln\rho_{ a_N}  \right|. $$
Applying the fact, that an element of the fundamental sequence is generally a partial product of the T-series $\rho_{ a_N} =  \left|a_N \right|  = \left| \prod_{n=1}^{N} r_n \right| =\prod_{n=1}^{N}\left| r_n \right| =\prod_{n=1}^{N}\rho_{ r_N} $, it gets rewritten as $$
l>\left|\ln\prod_{n=1}^{N} \rho_{ r_N} \right| =  \left|\sum_{n=1}^{N}\ln\rho_{ r_n} \right|.$$
Using the latter expression in (\ref{eq:brim}), one can obtain that \be\label{eq:dis}
 \exists  l>0 \in \mathbb R : \forall N,M \in \mathbb N : \left|\sum_{n=M}^{M+N}\ln\rho_{ r_n} \right| < 2l.\ee
Since $ \rho_{r_1} = |s + 2t \sin\mu|$ is a continuous function of $\mu$  and the torsion is $\pi$-irrational, \red{this somehow means that} (\ref{eq:dis}) is equivalent to $$
0 =\int_0^{2\pi} \ln (\rho_{r_1}) d\mu = \int_0^{2\pi} \frac12 \ln\left((x_s+ 2x_t \sin\mu)^2+(y_s + 2y_t \sin\mu)^2\right)d\mu 
.$$
\end{proof}

\begin{theorem}
For a fancy frond the brim is always empty and vice versa:
$$\rho_t>1 \Leftrightarrow B(t) = \emptyset.$$
\end{theorem}
\begin{proof}
\red{Needs to be written.}
\end{proof}

\begin{theorem}
For a cramped frond and a sleek torsion, the brim is represented by an ellipse on the complex plain with foci at $\pm t$ (the center at $0$) and semi-axes of $1\pm\rho_t^2$ (eccentricity is thus $2\rho_t/(1+\rho_t^2)$), namely:
% $$s \in B(t) \Leftrightarrow \left(\frac{x_t x_s +y_t y_s }{\rho_t(1+\rho_t^2)}\right)^2 + \left(\frac{x_t y_s  - x_s  y_t}{\rho_t(1-\rho_t^2)}\right)^2 = 1 ,$$
% or
$$s \in B(t) \Leftrightarrow \left(\frac{\rho_s\cos(\varphi_s-\varphi_t)}{1+\rho_t^2}\right)^2 + \left(\frac{\rho_s\sin(\varphi_s-\varphi_t)}{1-\rho_t^2}\right)^2 = 1 .$$
\end{theorem}
\begin{proof}
\red{No idea.}
\end{proof}

\section{The elemental roots}
\begin{definition}
Elemental root $\varphi_s$ of the fundamental sequence is such a whorl argument that there are $N$ smooth periodic functions 
$$ \big\{f_k(t): \mathbb \mathbb R \rightarrow \cmz\big\}_{k=1}^N 
,$$ called \blue{petals}, such that $$
\forall n \in \mathbb N: \exists k\le N \in \mathbb N,\,w \in \mathbb R: a_n = f_k(w) 
$$ and $$
\forall w \in \mathbb R,\, \forall \varepsilon >0  \in \mathbb R: \exists k\le N \in \mathbb N,\, n \in \mathbb N: |a_n - f_k(w)|<\varepsilon
.$$
The minimal number of petals $N$ needed to justify the criteria above is called its \blue{merosity}.  	
\end{definition}

\begin{theorem}
For a fancy frond and a sleek torsion $$\varphi_s = \frac{2\pi M}N, \qquad M \in \mathbb Z,\, N \in\mathbb N $$ is an elemental root of merosity $N$ if $\gcd(M,N)=1$.
\end{theorem}
\begin{proof}
\red{Needs to be written.}
\end{proof}

\end{document}
